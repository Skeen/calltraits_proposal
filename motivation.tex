Generic programming (decoupling of algorithm and type) has become the
recommended way for providing reusable code.

However, there are times, in which we cannot provide a truly generic solution,
and where we are forced to take type information into account, and hence we
need a generic way to query information about the instantiated type(s).

This is where traits come into play, traits provide us with a way to have type
specific information in a generic context. This is currently available for types
using the \verb|<type_traits>| header, however no similiar solution currently
exists for callable types.

The motivation for this proposal is to 'correct' this, by provide what the 
\verb|<type_traits>| header does for types, but for callable types.

Currently, when taking a callable type, by an unrestricted template, or when
storing it using the auto keyword, one looses absolutely all information about
the type, that is we're in an unknown state, about the callable type, and
there's currently no standard way for querying any of this information, if one
needs to.

Being able to query this kind of information, at compile-time will allow one
to generate definitions, and to support generic code, with type specific
behavior.

A pratictal toy example of this would be the following;
\begin{verbatim}
<code>
template <typename Callable, typename... Args>
struct converted_std_function_type_worker
{
    template <typename, typename>
    struct internal_struct;

    template <typename T, T... S>
    struct internal_struct<T, std::integer_sequence<T, S...>>
    {
        using std_function =
            std::function<typename std::return_type<Callable, Args...>(
                typename std::parameter_type<S, Callable, Args...>...)>;
    };
};

template <typename Callable, typename... Args>
using converted_std_function_type =
    typename converted_std_function_type_worker<Callable, Args...>::
        template internal_struct<size_t,
            std::make_integer_sequence<size_t,
                std::arity_integral<Callable, Args...>::value>>::
                    std_function;

template <typename Callable, typename... Args>
auto make_stdfunction(Callable c, Args... args)
    -> converted_std_function_type<Callable, Args...>
{ return c; }
</code>
\end{verbatim}
Note: \verb|std::integer_sequence<T, S...>| and friends refer to the concepts from
\href{http://www.open-std.org/jtc1/sc22/wg21/docs/papers/2013/n3493.html}{N3493}.
\noindent
The above code allows one to convert any callable type to a std::function, by
extracting the required template information from the callable type at compile time.

Usage code may look like;
\begin{verbatim}
<code>
auto func = make_stdfunction(&function);
static_assert(std::is_same<decltype(func),
              std::function<int(int)>>::value, "");

auto lambda = make_stdfunction([](int){ return 'C'; });
static_assert(std::is_same<decltype(lambda),
              std::function<char(int)>>::value, "");
</code>
\end{verbatim}
The above is of course just a simple toy example, for the purpose of this text.
Plenty of other, possibly more useful examples can be thought up. 
