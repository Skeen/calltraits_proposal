\subsection{Interface}
I'm open to adding more traits to the interface, assuming they are implementable
in a pure library solution.

\subsubsection{Arity}
\begin{itemize}
\item[(1)] Return the arity (e.g. number of arguments) for a given callable type.
\item[(2)] Return whether the callable type has the queried number of arguments.
\end{itemize}
\begin{verbatim}
// 1
template<typename Callable, typename... Args>
constexpr size_t arity = ...;

// 2
template <typename Callable, typename... Args>
constexpr bool is_nullary = (arity<Callable, Args...> == 0);
template <typename Callable, typename... Args>
constexpr bool is_unary = (arity<Callable, Args...> == 1);
template <typename Callable, typename... Args>
constexpr bool is_binary = (arity<Callable, Args...> == 2);
template <typename Callable, typename... Args>
constexpr bool is_ternary = (arity<Callable, Args...> == 3);
\end{verbatim}
Note; In the case of varying arguments (C-style ellipsis), std::arity returns the
number of actual arguments.
Note; As default arguments are not a part of a function type (8.3.6), these will
always be 'ignored', in terms of std::arity.
Note; For member functions, this will return the actual number of arguments, 
that is without the implicit object parameter, as this is handled seperately by
the traits; 'is\_member\_function', and 'implicit\_object\_type'.

\subsubsection{Has Varying Arguments}
Return whether or not the callable type, accepts a varying number of arguments
(e.g. C style variadics, void func(...)).
\begin{verbatim}
template <typename Callable, typename... Args>
constexpr bool has_varying_arguments = ...;
\end{verbatim}
Note; Has nothing to do with variadic templates.

\subsubsection{Is function pointer}
Return whether the callable type is a function pointer.
\begin{verbatim}
template<typename Callable>
constexpr bool is_function_pointer = ...;
\end{verbatim}

\subsubsection{Is const function}
Return whether the given member function, is declared const.
\begin{verbatim}
template <typename Callable, typename... Args>
constexpr bool is_const_function = ...;
\end{verbatim}
Note; Only ever non-false for non-static member functions.

\subsubsection{Is member function}
Return whether the given callable type, is a member function.
\begin{verbatim}
template<typename Callable, typename... Args>
constexpr bool is_member_function = ...;
\end{verbatim}
Note; This currently returns false for static functions, as we're not able to
disambiguate static functions from free functions.
The only way to disambiguate would be a compiler supported solution.

\subsubsection{Is static or free function}
Return whether the given callable type, is a static or free function.
\begin{verbatim}
template<typename Callable, typename... Args>
constexpr bool is_static_member_or_free_function = ...;
\end{verbatim}
Note; See note in 'Is member function'. - Currently returns the complete
opposite of is\_member\_function

\subsubsection{Implicit object type}
Return the type of the implicit object parameter for a given member function.
\begin{verbatim}
template<typename Callable, typename... Args>
using implicit_object_type = ...;
\end{verbatim}
Note; This is only defined for callable's returning true for;
'Is member function'

\subsubsection{Parameter type}
Return the type of the i'th parameter for a given callable type.
\begin{verbatim}
template<int I, typename Callable, typename... Args>
using parameter_type = ...;
\end{verbatim}
Note; Is only defined for; $0 < I < std::arity<Callable>$.
TODO; Figure if it's possible in a pure library solution, to retain qualifiers. 

\subsubsection{Return type}
Return the return type for a given callable type.
\begin{verbatim}
template<typename Callable, typename... Args>
using return_type = ...;
\end{verbatim}

\subsection{Usage}
The above traits can be used, in a few ways;
\subsubsection{Functions}
Using decltype(...), for declared functions;
\begin{verbatim}
void function(int, double);
std::return_type<decltype(function)> // void
std::parameter_type<0, decltype(function)> // int
\end{verbatim}
Using decltype(\&...), for declared functions;
\begin{verbatim}
void function(int, double);
std::return_type<decltype(&function)> // void
std::parameter_type<0, decltype(&function)> // int
\end{verbatim}

Using template specifiction, for non-declared functions;
(which the above case, maps down to after decltype)
\begin{verbatim}
std::arity<void(...)> // 0
std::has_varying_arguments<void(...)> // true
\end{verbatim}

\subsubsection{Functors}
Passing functor types directly;
\begin{verbatim}
struct functor
{
    void operator()(int);
};
std::return_type<functor> // void
std::implicit_object_type<functor> // functor
\end{verbatim}
Passing functor types directly, but specifying the overload;
\begin{verbatim}
struct variadic_template_functor
{
    template <typename... Ts>
    void operator()(Ts..., ...);
};
std::arity<variadic_template_functor, int, double> // 2
std::has_varying_arguments<variadic_template_functor, int, double> // true
std::parameter_type<1, variadic_template_functor, int, double> // double
\end{verbatim}

\subsubsection{Member functions}
All class methods, are essencially function pointers, and hence will be called
the same way, just with a prefix;
\begin{verbatim}
struct klass
{
    void function(int);
    void const_function(int) const;
    static void static_function(int);
};
std::is_member_function<decltype(&klass::function)> // true
std::is_const_function<decltype(&klass::const_function)> // true
std::is_static_member_or_free_function<decltype(&klass::static_function)> // true
\end{verbatim}
Do note however, that for member functions (e.g. non-static functions defined on
the class), we need the '\&', while it can be left out for static functions;
\begin{verbatim}
std::is_static_member_or_free_function<decltype(klass::static_function)> // true
\end{verbatim}
The reason for why static functions are treated differently, is because these
are essentially free functions, and hence subject to the C rules for functions
and function pointers. 

Note; Templated and overloaded methods, are handled by explicitly specifying the
overload.

\subsubsection{Lambdas}
Lambdas are essentially functors, and hence behave as these, however as lambdas
are constructed as objects of an anonymous type, we'll need decltype(...) to get
the underlaying type.
\begin{verbatim}
const auto lambda = [](int i){ return i; };
std::is_member_function<decltype(lambda)> // true
std::is_const_function<decltype(lambda)> // true
\end{verbatim}
Note; C++14 Polymorphic lambdas will be supported by means of explicitly specifying the
overload.

\subsubsection{Type overloading 'operator()'}
For all other types, which overload the call operator, we have behavior as in
the functor case. For instance with std::function;
\begin{verbatim}
std::is_member_function<std::function<int(int)>> // true
std::is_same<std::return_type<std::function<int(int)>> // int
std::is_same<std::parameter_type<0, std::function<int(int)>> // int
\end{verbatim}

\subsection{Implementatation}
The proposal is concerned with the interface only, and hence compiler and
standard library vendors are free to do their implementation in any way they'd
like, be it compiler argumented or a pure library solution.

A pure library implementation has been developed, and is available at;
\href{}{SourceForge}
TODO: INSERT LINK

Note; This implementation currently makes use of a few C++1y features, however
the library can be implemented without these.
